%% start of file `template.tex'.
%% Copyright 2006-2015 Xavier Danaux (xdanaux@gmail.com).
%
% This work may be distributed and/or modified under the
% conditions of the LaTeX Project Public License version 1.3c,
% available at http://www.latex-project.org/lppl/.


\documentclass[11pt,a4paper,sans]{moderncv}        % possible options include font size ('10pt', '11pt' and '12pt'), paper size ('a4paper', 'letterpaper', 'a5paper', 'legalpaper', 'executivepaper' and 'landscape') and font family ('sans' and 'roman')

% moderncv themes
\moderncvstyle{banking}                             % style options are 'casual' (default), 'classic', 'banking', 'oldstyle' and 'fancy'
\moderncvcolor{burgundy}                               % color options 'black', 'blue' (default), 'burgundy', 'green', 'grey', 'orange', 'purple' and 'red'
%\renewcommand{\familydefault}{\sfdefault}         % to set the default font; use '\sfdefault' for the default sans serif font, '\rmdefault' for the default roman one, or any tex font name
\nopagenumbers{}                                  % uncomment to suppress automatic page numbering for CVs longer than one page

% character encoding
%\usepackage[utf8]{inputenc}                       % if you are not using xelatex ou lualatex, replace by the encoding you are using
%\usepackage{CJKutf8}                              % if you need to use CJK to typeset your resume in Chinese, Japanese or Korean

% adjust the page margins
\usepackage[scale=0.75]{geometry}
%\setlength{\hintscolumnwidth}{3cm}                % if you want to change the width of the column with the dates
%\setlength{\makecvheadnamewidth}{10cm}            % for the 'classic' style, if you want to force the width allocated to your name and avoid line breaks. be careful though, the length is normally calculated to avoid any overlap with your personal info; use this at your own typographical risks...

% personal data
\name{Franck}{Parat}
\title{Embedded Software Engineer}                               % optional, remove / comment the line if not wanted
%\address{street and number}{postcode city}{country}% optional, remove / comment the line if not wanted; the "postcode city" and "country" arguments can be omitted or provided empty
\phone[mobile]{+86~135~6420~1140}                   % optional, remove / comment the line if not wanted; the optional "type" of the phone can be "mobile" (default), "fixed" or "fax"
%\phone[fixed]{+2~(345)~678~901}
%\phone[fax]{+3~(456)~789~012}
\email{franck.parat@gmail.com}                               % optional, remove / comment the line if not wanted
%\homepage{www.johndoe.com}                         % optional, remove / comment the line if not wanted
%\social[linkedin]{john.doe}                        % optional, remove / comment the line if not wanted
%\social[xing]{john\_doe}                           % optional, remove / comment the line if not wanted
%\social[twitter]{jdoe}                             % optional, remove / comment the line if not wanted
%\social[github]{jdoe}                              % optional, remove / comment the line if not wanted
%\social[gitlab]{jdoe}                              % optional, remove / comment the line if not wanted
%\social[skype]{jdoe}                               % optional, remove / comment the line if not wanted
%\extrainfo{additional information}                 % optional, remove / comment the line if not wanted
\photo[64pt][0.4pt]{picture}                       % optional, remove / comment the line if not wanted; '64pt' is the height the picture must be resized to, 0.4pt is the thickness of the frame around it (put it to 0pt for no frame) and 'picture' is the name of the picture file
%\quote{Some quote}                                 % optional, remove / comment the line if not wanted

% bibliography adjustements (only useful if you make citations in your resume, or print a list of publications using BibTeX)
%   to show numerical labels in the bibliography (default is to show no labels)
%\makeatletter\renewcommand*{\bibliographyitemlabel}{\@biblabel{\arabic{enumiv}}}\makeatother
\renewcommand*{\bibliographyitemlabel}{[\arabic{enumiv}]}
%   to redefine the bibliography heading string ("Publications")
%\renewcommand{\refname}{Articles}

% bibliography with mutiple entries
%\usepackage{multibib}
%\newcites{book,misc}{{Books},{Others}}

% Adjust the font of the name and title to avoid wrap
\renewcommand*{\namefont}{\huge\bfseries\upshape}
\renewcommand*{\titlefont}{\LARGE\mdseries\upshape}

%----------------------------------------------------------------------------------
%            content
%----------------------------------------------------------------------------------
\begin{document}
%-----       resume       ---------------------------------------------------------
\makecvtitle

\section{Engineering skills}
\cvitem
  {Key programming skills}
  {C (bare metal), Python 2/3, VHDL (FPGA)}
\cvitem
  {Embedded engineering}
  {UART, SPI, CAN, LIN, K-Line, I\textsuperscript{2}C, ADC, DAC, timers, interrupts, HAL}
\cvitem
  {Hardware}
  {Debugging, usage of oscilloscope, soldering, reading of schematics and datasheets}
\cvitem
  {Also}
  {C++, Assembly, Java, Javascript, Typescript (Angular), Batch, Latex}
\cvitem
  {Tools}
  {Git, Subversion, Ceedling (test framework for C), Sphinx (documentation), "Unix" utilities}

\section{Experience}
\cventry
  {2014--\the\year}
  {Automotive Testing Industry}
  {ART Logics}
  {Shanghai}
  {Embedded Software Engineer}
  {
    Worked on the logic implementation of key products of the company.
    \begin{itemize}%
    \item Maintainer of the TCU/CCU firmwares, core products of the company (SPC56, PIC24):
      \begin{itemize}%
      \item Design and implementation of new features.
      \item Port the code to the sibling product CCU100, and the successor TCU110;
      \item Refactoring of the code for supporting multiple products with a single codebase.
      \item Integrated unit testing and documentation generation in the workflow.
      \end{itemize}
    \item Development and maintenance of FPGA "firmwares" (Xilinx Spartan 6):
      \begin{itemize}%
      \item Bug fixes and stabilization of boards in production;
      \item Design and development of several boards.
      \end{itemize}
    \item Design and development of smaller products based on PIC24, with LIN and UART interfaces.
    \item Creation of a desktop application for UART or Ethernet communication. Based on Python and wxWidgets.
    \item Development of a desktop application for the calibration of several products. Based on Python and Tkinter.
    \end{itemize}
  }

\section{Education}
\cventry
  {2013--2014}
  {Exchange program}
  {National University of Singapore}
  {Singapore}
  {}
  {Exchange program for 2 semesters (Embedded Systems, Computer Science, Multimedia)}  % arguments 3 to 6 can be left empty
\cventry
  {2011--2014}
  {Engineer degree (master level)}
  {ENSEA}
  {Cergy}
  {}
  {Graduate School of Electrical Engineering and Computer Science}
\cventry
  {2009--2011}
  {"Classe Préparatoire aux Grandes Ecoles"}
  {Lycee Saint-Louis}
  {Paris}
  {}
  {MPSI, PSI: Advanced Mathematics, Physics and Engineering Science}


\section{Languages}
\cvdoubleitem
  {French}{Mother tongue}
  {Chinese (mandarin)}{Conversational}
\cvdoubleitem
  {English}{Fluent. TOEIC: 930, TOEFL: 93}
  {}{}


\end{document}
