%% start of file `template.tex'.
%% Copyright 2006-2015 Xavier Danaux (xdanaux@gmail.com).
%
% This work may be distributed and/or modified under the
% conditions of the LaTeX Project Public License version 1.3c,
% available at http://www.latex-project.org/lppl/.


\documentclass[11pt,a4paper,sans]{moderncv}        % possible options include font size ('10pt', '11pt' and '12pt'), paper size ('a4paper', 'letterpaper', 'a5paper', 'legalpaper', 'executivepaper' and 'landscape') and font family ('sans' and 'roman')

% moderncv themes
\moderncvstyle{banking}                             % style options are 'casual' (default), 'classic', 'banking', 'oldstyle' and 'fancy'
\moderncvcolor{burgundy}                               % color options 'black', 'blue' (default), 'burgundy', 'green', 'grey', 'orange', 'purple' and 'red'
%\renewcommand{\familydefault}{\sfdefault}         % to set the default font; use '\sfdefault' for the default sans serif font, '\rmdefault' for the default roman one, or any tex font name
\nopagenumbers{}                                  % uncomment to suppress automatic page numbering for CVs longer than one page

% character encoding
%\usepackage[utf8]{inputenc}                       % if you are not using xelatex ou lualatex, replace by the encoding you are using
%\usepackage{CJKutf8}                              % if you need to use CJK to typeset your resume in Chinese, Japanese or Korean

% adjust the page margins
\usepackage[scale=0.75]{geometry}
%\setlength{\hintscolumnwidth}{3cm}                % if you want to change the width of the column with the dates
%\setlength{\makecvheadnamewidth}{10cm}            % for the 'classic' style, if you want to force the width allocated to your name and avoid line breaks. be careful though, the length is normally calculated to avoid any overlap with your personal info; use this at your own typographical risks...

% personal data
\name{Franck}{Parat}
\title{Embedded Software Engineer}
%\address{street and number}{postcode city}{country}
\phone[mobile]{+33~6~70~68~84~31}
%\phone[fixed]{+2~(345)~678~901}
\email{franck.parat@gmail.com}
%\homepage{www.johndoe.com}
\social[linkedin]{franck-parat}
%\social[xing]{john\_doe}
%\social[twitter]{jdoe}
\social[github]{fparat}
%\social[gitlab]{jdoe}
\social[skype]{f.parat}
%\extrainfo{additional information}
\photo[64pt][0.4pt]{picture}
%\quote{Some quote}

% bibliography adjustements (only useful if you make citations in your resume, or print a list of publications using BibTeX)
%   to show numerical labels in the bibliography (default is to show no labels)
%\makeatletter\renewcommand*{\bibliographyitemlabel}{\@biblabel{\arabic{enumiv}}}\makeatother
\renewcommand*{\bibliographyitemlabel}{[\arabic{enumiv}]}
%   to redefine the bibliography heading string ("Publications")
%\renewcommand{\refname}{Articles}

% bibliography with mutiple entries
%\usepackage{multibib}
%\newcites{book,misc}{{Books},{Others}}

% Adjust the font of the name and title to avoid wrap
\renewcommand*{\namefont}{\huge\bfseries\upshape}
\renewcommand*{\titlefont}{\LARGE\mdseries\upshape}

% Date of last update in the footer
\rfoot{\footnotesize\textcolor{color2}{\textit{Last update: \today}}}

%----------------------------------------------------------------------------------
%            content
%----------------------------------------------------------------------------------
\begin{document}
%-----       resume       ---------------------------------------------------------
\makecvtitle

\section{Skills}
\cvitem
  {Key programming skills}
  {C, Python 2/3, Rust, VHDL (FPGA)}
\cvitem
  {Embedded engineering}
  {UART, SPI, CAN (FD), LIN, K-Line, I\textsuperscript{2}C, Ethernet, HAL, Drivers}
\cvitem
  {Hardware}
  {Debugging, usage of oscilloscope, soldering, reading of schematics and datasheets}
\cvitem
  {Also}
  {C++, Linux, Assembly, Bash, Batch, Docker, Latex}
\cvitem
  {Tools}
  {Versionning (Git, SVN), Testing (Ceedling, pytest), Documentation (Sphinx), "Unix" utilities}

\section{Experience}
\cventry
  {2014--\the\year}
  {Automotive Testing Industry}
  {ART Logics}
  {Shanghai}
  {Embedded Software Engineer}
  {
    Worked on the logic implementation of key products of the company.
    \begin{itemize}%
    \item Responsible of the embedded software of a generic test controller. (SPC56/58 (PowerPC), PIC24):
      \begin{itemize}%
      \item Design, implement, test, and document new features,
      \item Refactored the code with a strict hardware abstraction layer for portability accross different hardware,
      \item Integrated automatic unit testing and documentation generation in the development workflow.
      \end{itemize}
    \item Develop and maintain FPGA logic implementations (Xilinx Spartan 6):
      \begin{itemize}%
      \item Bug fixes and stabilization of boards in production,
      \item Logic design and implementation for several new products.
      \end{itemize}
    \item Developed the firmware of smaller products based on PIC24, with LIN and UART interfaces.
    \item Created a GUI tool for controlling the products with UART/Ethernet. Based on Python and wxWidgets.
    \item Developed a desktop application for the calibration of several products. Based on Python and Tkinter.
    \end{itemize}
  }

\section{Education}
\cventry
  {2013--2014}
  {Exchange program}
  {\href{https://en.wikipedia.org/wiki/National_University_of_Singapore}{National University of Singapore (NUS)}}
  {Singapore}
  {}
  {Exchange program for 2 semesters (Embedded Systems, Computer Science, Multimedia)}  % arguments 3 to 6 can be left empty
\cventry
  {2011--2014}
  {Engineer degree (master level)}
  {\href
    {https://en.wikipedia.org/wiki/\%C3\%89cole_nationale_sup\%C3\%A9rieure_de_l\%27\%C3\%A9lectronique_et_de_ses_applications}
    {École Nationale Supérieure de l'Électronique et de ses Applications (ENSEA)}
  }
  {Cergy}
  {}
  {Graduate School of Electrical Engineering and Computer Science}
\cventry
  {2009--2011}
  {"Classe Préparatoire aux Grandes Ecoles"}
  {\href{https://en.wikipedia.org/wiki/Lyc\%C3\%A9e_Saint-Louis}{Lycée Saint-Louis}}
  {Paris}
  {}
  {MPSI, PSI: Advanced Mathematics, Physics and Engineering Science}

\section{Languages}
\cvdoubleitem
  {French}{Mother tongue}
  {Chinese (mandarin)}{Conversational}
\cvdoubleitem
  {English}{Fluent. TOEIC: 930, TOEFL: 93}
  {}{}

%\vspace*{\fill}
%\rightline{\textcolor{color2}{\textit{Last update: \today}}}

\end{document}
